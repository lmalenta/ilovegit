\documentclass{article}
\usepackage{graphicx} % Required for inserting images

\title{Primary vertex timing reconstruction with the LHCb Ring Imaging Cherenkov detectors }
% \author{Lorenzo Malentacca}
\date{}

\begin{document}

\maketitle

During LHC High-Luminosity phase, the LHCb RICH detector will face challenges due to increased particle multiplicity and high occupancy. Introducing sub-100ps time information becomes crucial for maintaining excellent particle identification (PID) performance. The LHCb RICH collaboration plans to anticipate the introduction of timing through an enhancement program during the third LHC Long Shutdown. In the RICH detector, Cherenkov photons from a track arrive nearly simultaneously at the detector plane, allowing precise hit time prediction. The RICH reconstruction algorithm computes track and photon time-of-flight and estimates where photons are expected on the photodetector plane. Determining the primary vertex time (PV T$_0$) is crucial in predicting the time of arrival of photons on the photodetector plane. Adding time information allows applying a software time gate around the predicted time per track to enhance signal-to-background ratio and PID performance. This contribution describes how to estimate the PV T$_0$ using RICH information only, a novel approach for LHCb. The proposed algorithm computes a reconstructed PV time for every photon from hit time and tracking information. The PV T$_0$ is extracted by averaging this reconstructed time for all photons belonging to the PV. The challenge lies in correctly associate photons to their PV, which is a two-step process: PV-track and track-photon associations, both presenting inefficiencies. Results compare the estimated PV time resolution with Monte Carlo simulations. This contribution aims to describe the integration of fast-timing in the RICH detector, illustrating the impact of the PV time estimation method on PID performance.



\end{document}
